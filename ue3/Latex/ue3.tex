\documentclass[11pt, a4paper]{scrartcl}
\usepackage[ngerman]{babel}
%\usepackage[T1]{fontenc}
\usepackage[utf8]{inputenc}
\usepackage{amsmath}
\usepackage{amssymb}
\usepackage{amsthm}
\usepackage{graphicx}
\usepackage{listings}
\usepackage{booktabs}
\usepackage{float}
\usepackage{pdfpages}
%
\begin{document}
\lstset{basicstyle=\small,
		 inputencoding=latin1,
		stringstyle=\ttfamily,
		identifierstyle=,
		showstringspaces=false,
		language=c,
		frame=trBL}
%
\subsection*{TI III WS 2013, Fr. 12-14}
\section*{Lösung Übungsblatt 3}
\textbf{Christoph van Heteren-Frese (Matr.-Nr.: 4465677), Julien  Stengel } \\%(Matr.-Nr.: 4567553)}\\
Tutor: Ruhland, eingereicht am \today\\
\hrule
%
\section*{Aufgabe 4}
\subsection*{a)}
\subsubsection*{Wieso haben die Variablen \texttt{a} und \texttt{b} zu unterschiedlichen Zeitpunkten verschiedene Werte?}
Die Variablen \texttt{a} und \texttt{b} wurden mehrfach deklariert; zum einen \textit{global} in den Zeilen 3 und 4, sowie \textit{lokal} in den Funktionen \texttt{diff} (Zeile 14 und 15) und \texttt{main} (Zeile 23 und 24). Die Ausgabe des Programms besteht aus fünf Zeilen und wird durch mehrere Aufrufe der Bibliotheksfunktion \texttt{printf} erzeugt:
\begin{enumerate}
\item[(1)] Zeile eins der Ausgabe wird durch den Aufruf von \texttt{printf} in Zeile 27 erzeugt, der die Werte der zur Funktion \texttt{main} \textit{lokalen} Variablen \texttt{a} und \texttt{b} ausgibt
\item[(2)] Zeile zwei der Ausgabe wird durch den Aufruf von \texttt{printf} innerhalb der Funktion \texttt{summe} (Zeile 8) erzeugt, die die \textit{global} deklarierten Variablen \texttt{a} und \texttt{b} ausgibt 
\item[(3)] Zeile drei der Ausgabe wird durch den Aufruf von \texttt{printf} in Zeile 28 erzeugt. Die \textit{lokalen} Variablen \texttt{a} und \texttt{b} werden als Parameter an die Funktion \texttt{summe} übergeben, die die Summe der beiden Werte zurück gibt.
\item[(4)] Zeile vier der Ausgabe wird durch den Aufruf von \texttt{printf} in Zeile 17 innerhalb der Funktion \texttt{diff} erzeugt. Die Werte der zur Funktion \texttt{diff} \textit{lokalen} Variablen \texttt{a} und \texttt{b} werden hier ausgegeben. 
\item[(5)] Zeile fünf der Ausgabe wird durch den Aufruf von \texttt{printf} in Zeile 31 erzeugt, der den Wert der zur Funktion \texttt{main} \textit{lokalen} Variable \texttt{c} ausgibt. Die Speicheradresse der Variable \texttt{c} ist zuvor durch den Aufruf in Zeile 30 als Parameter an die Funktion \texttt{diff} übergeben und dort in Folge der Zuweisung in Zeile 18 überschrieben worden. 
\end{enumerate} 
\subsubsection*{Welchen Rückgabewert hat die Funktion diff()?}
Die Funktion \texttt{diff} ist vom Typ \texttt{void} und hat somit keinen Rückgabewert.
\newpage
\subsection*{b)}
\lstinputlisting{../u3_4b.c}
\newpage
\subsection*{c)}
\lstinputlisting{../u3_4b_2.c}
\newpage
%\includepdf[pages={1}]{blatt1.pdf}
%\includepdf[pages={1}]{blatt2.pdf}
\end{document}
