\documentclass[11pt, a4paper]{scrartcl}
\usepackage[ngerman]{babel}
%\usepackage[T1]{fontenc}
\usepackage[utf8]{inputenc}
\usepackage{amsmath}
\usepackage{amssymb}
\usepackage{amsthm}
\usepackage{graphicx}
\usepackage{listings}
\usepackage{booktabs}
\usepackage{float}

\begin{document}
\onecolumn
\subsection*{TI III WS 2013, Fr. 12-14}
\section*{Lösung Übungsblatt 1}
\textbf{Christoph van Heteren-Frese (Matr.-Nr.: 4465677)} \\ % Sven Wildermann (Matr.-Nr.: 4567553)}\\
Tutor: Ruhland, eingereicht am \today\\
\hrule

\section*{Aufgabe 1}
\subsection*{a)}
\begin{figure}[H]
\center
\begin{tabular}{p{12cm}}
\toprule
\textit{Hardware} \\
\midrule 
CPU, Transistor, Festplatte, Spei\-cher, CD-ROM \\ 
\bottomrule \addlinespace
\textit{Betriebssystem} \\
\midrule
Scheduler, Assemb\-ler, Lader, Speichermanage\-ment,~Da\-tei\-sys\-tem, Makro\-pro\-zess\-or, Bios, Cache, Netz\-werk\-un\-ter\-stütz\-ung, Interprozesskommunikation\\
\bottomrule \addlinespace
\textit{Anwendungs\-pro\-gram\-me} \\
\midrule
Datenbanksystem, Texteditor, Graphikprogramm, Compil\-er, Bibliotheken, Linker, X-Windows, Text\-ver\-ar\-bei\-tungs\-pro\-gramm, Gerätemanagement, WWW-Browser\\
\bottomrule \addlinespace
\textit{Nutzer} \\
\midrule
Student/Studentin, ProffesorIn, Sekretär\\
\bottomrule
\end{tabular}
\end{figure}
\subsection*{b)}
Dem Betriebssystem (englisch: \textit{operating system}, kurz OS) kommt hier eine Vermittlerrolle zu. Es ist dafür veranwortlich eine funktionierende Arbeitsumgebung be\-reit\-zustellen, so dass Anwwendungen (Software) auf einem Computersystem \glqq betrieben\grqq, also genutzt werden können. Das OS teillt dabei einer Anwendung u.a. (virtuellen) Speicher zu und sorgt für die Zuteilung von Prozessorzeit.
\section*{Aufgabe 2}
\begin{itemize}
 \item Der Befehlssatz (englisch: \textbf{Instruction Set})
umfasst die Menge der Maschinenbefehle eines Prozessors,
die für den Programmierer sichtbar ist. Die ISA
(Instruction Set Architecture) stellt damit eine Schnittstelle zwischen Soft-
und  Hardware dar. 
\item Ein \textbf{Interrupt} bezeichnet die kurzfristige Unterbrechnung
der sequentiellen Programmausführung auf Grund einer dem Prozess
fremden Ursache mit dem Zweck eine andere, zeitkritische (und kurze) 
Verarbeitung durchzuführen. Nach der Unterbrechnung wird der
vorherige Prozess fortgeführt, wo er unterbrochen wurde. 
\item Ein \textbf{Prozess} beschreibt eine Einheit mit ausführbarer
Befehlssequenz, aktueller Statusinformation und zugeordnetem 
Speicherbereich (Register, shared memory etc.). Der Begriff
Task ist mit dem Begriff Prozess äquivalent.
\item Eine \textbf{Datei} ist eine vom Dateisystem geleitetes, 
immer währendes Objekt mit dem Zweck der Langzeitspeicherung. 
Diese wird im ``zweiten Speicher'' (z.B. USB Speicher, Festplatte)
gespeichert und ähnelt dem virtuellen Speicher dahin gehend,
dass nicht die physische Adresse, sondern nur der Dateiname
dem Benutzer bekannt ist. 
\item Ein \textbf{Systemaufruf} (engl. Systemcall oder Syscall)
ist eine Methode, mit der Benutzeranwendungen Zugang zu System
Services (=vom OS bereitgestellte Funktionalitäten) erhalten.
\item Der Begriff \textbf{Multitasking} beschreibt die Fähigkeit
eines Betriebssystems, mehrere Tasks (Prozesse) nebenläufig auszuführen.
Hierfür werden die verschiedenen Prozesse immer wieder abwechselnd 
ausgeführt, so dass für den Benutzer der Eindruck der gleichzeitigen 
Ausführung entsteht (wenn Anzahl der Tasks $>$ Anzahl der CPU-Kerne).
\end{itemize}
\section*{Aufgabe 3}
Während in einem \textbf{Mikrokernel} nur grundlegende
Funktionen (zur Speicher- und Pro\-zess\-ver\-waltung; 
Grundfunktionen zur Kommunikation) zur Verfügung gestellt
werden, sind in einem \textbf{monolithischem} Kernel 
neben diesen auch noch Treiber für die 
Hardwarekomponenten und weitere Funktionen implementiert.
\\ \\
\textbf{Monolithischer Kernel}\\
Vorteile: Weniger Kontextsprünge; keine teure Kommunikation\\
Nachteil: Komplikationen beim Austausch von Funktionen\\ \\
\textbf{Mikrokernel}\\ 
Vorteile: präzise Schnittstelle, weniger Komplexität, klare 
Strukturen\\
Nachteile: Mehr Synchronisationsaufwand, weniger Geschwindigkeit
\newpage
\section*{Aufgabe 4}
\subsection*{a)}
\lstset{basicstyle=\small,
		 inputencoding=latin1,
		stringstyle=\ttfamily,
		identifierstyle=,
		showstringspaces=false,
		language=c,
		frame=trBL}
\lstinputlisting{fahrenheit.c}
\subsection*{b)}
\lstinputlisting{fahrenheit_tab.c}
\end{document}
